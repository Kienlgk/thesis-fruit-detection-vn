\chapter{GIỚI THIỆU}
\label{introduction}
\section{Giới thiệu bài toán}
Trí tuệ nhân tạo được khởi xướng từ giữa thế kỉ XX, lúc đó một nhà toán học trẻ người Anh tên là Alan Turing, hiện nay được coi là ông tổ của lĩnh vực trí tuệ nhân tạo, đã đề xuất ra ý tưởng làm cho máy móc có thể dựa vào dữ liệu sẵn có để được ra quyết định giống như con người \footnote{Source: \url{http://sitn.hms.harvard.edu/flash/2017/history-artificial-intelligence/}}. Vào thời ấy, máy móc để tính toán là một thứ hết sức xa xỉ chỉ có ở những công ty và trường đại học lớn, nên sự phát triển của lĩnh vực trí tuệ nhân tạo chỉ đạt được thành tựu nhất định và chưa tạo ra được sức ảnh hưởng lớn. Tuy nhiên, trong những năm gần đây, nhờ sự phát triển của công nghệ và sức mạnh phần cứng máy tính, các ứng dụng của công nghệ trí tuệ nhân tạo ngày càng phát triển và đã và đang được áp dụng rất nhiều vào thực tiễn, đặc biệt là để giải quyết một cách thông minh những vấn đề mà con người khó mà giải quyết nhanh chóng được. Trí tuệ nhân tạo với sự phát triển của kĩ thuật học máy (Machine Learning) đã góp phần hé mở đáp án cho những bài toán mà trước đây được coi là không thể giải được. Nhờ áp dụng học máy, các công ty lớn hiện nay như Google, Microsoft, Facebook có thể tạo ra nhiều phần mềm thông minh, hỗ trợ tốt cho con người trong nhiều lĩnh vực từ giải trí đến công việc. Học máy là một phương pháp để giải quyết các bài toán ra quyết định một cách linh hoạt, nó học những dữ liệu trường hợp đã có, từ đó có thể đưa ra phán đoán, quyết định cho những trường hợp chưa từng gặp với một độ chính xác chấp nhận được. Học máy hiện nay được áp dụng cho nhiều lĩnh vực như: xử lí ngôn ngữ tự nhiên, quá trình khai phá dữ liệu, bảo mật (phát hiện các giao dịch gian lận, phân loại thư rác, ...). Trong đó, một lĩnh vực không thể không nhắc đến bởi vì tầm quan trọng của nó đó chính là thị giác máy tính (computer vision).

Thị giác máy tính là một trong các lĩnh vực để giải quyết bài toán phân loại, nhận diện hình ảnh, được nhiều nhà khoa học và doanh nghiệp và cộng đồng quan tâm tới, đặc biệt là những bài toán áp dụng vào thực tiễn, nhất là trong lĩnh vực nông nghiệp. Trước đây, để xác định trái cây đã đến lúc thu hoạch được chưa, cây trồng này đang mắc phải ôn dịch có hại, mầm bệnh nào hoặc xác định năng suất cây ăn trái, đa phần người nông dân sử dụng kinh nghiệm và trực giác của mình. Tuy rằng kinh nghiệm canh tác của họ tốt, tuy nhiên việc nhận biết, phát hiện, đánh giá năng suất trên một khu vườn quá rộng sẽ rất mất công sức và nhân lực nếu chỉ thực hiện một cách thủ công. Giờ đây, bằng các kĩ thuật xử lí ảnh hiện đại, chỉ với những bức ảnh được chụp lại từ vườn cây, các thông tin tình trạng cây, năng suất, nguy cơ mắc bệnh sẽ được phân tích và đưa ra kết quả với xác suất chính xác cao và nhanh chóng, từ đó đánh giá được tình trạng, năng suất của cây trồng và đưa ra phương án chuyên môn thích hợp để tối ưu năng suất cho vườn cây. Điều này góp phần cải thiện đáng kể năng suất trong trồng trọt của người nông dân và khẳng định vai trò vô cùng quan trọng của việc áp dụng những công nghệ, phương pháp thông minh vào thực tiễn nông nghiệp trong nước và thế giới.

Thực tế trên thế giới hiện nay, thu hoạch cây ăn trái bằng robot tự động không còn là vấn đề xa lạ. Nhờ áp dụng công nghệ tự động hóa như vậy cùng với các máy móc hiện đại, chúng ta thu được các loại dữ liệu đa dạng để đánh giá năng suất của cây trồng trong vườn và ước tính được biểu đồ năng suất, đó là một vấn đề phổ biến nhưng lại hết sức quan trọng đối với nông nghiệp. Dữ liệu thu thập được dễ dàng nhất đó chính là hình ảnh của các cây trồng, được chụp từ những máy ảnh được gắn trên các thiết bị tự động hóa sử dụng trong khu vườn. Việc thu thập hình ảnh về cây ăn trái trên vườn được thực hiện bằng cách sử dụng các phương tiện di chuyển tự động trên khu vực trồng trọt và chụp ảnh với mục đích là đưa gần như toàn bộ cây vào ảnh đồng thời qua đó cung cấp dữ liệu cho các bài toán thực tiễn trong nông nghiệp. Đây cũng là một nguồn dữ liệu khả thi cho các giải thuật trí tuệ nhân tạo sử dụng để huấn luyện bộ học của nó trở nên thông minh và nhanh nhạy.

Trong lĩnh vực thị giác máy tính, nhờ sự phát triển của mạng nơ-ron học sâu (Deep Neural Networks - DNNs) ứng dụng cho bài toán phát hiện vật thể, việc học đặc trưng của ảnh đã trở nên dễ dàng và hiệu quả hơn, từ đó tiến tới phát triển các phương pháp phân loại, nhận diện vật thể trong hình ảnh. Đối với bài toán đánh giá năng suất cây ăn trái, việc nhận diện được vật thể trái cây trong ảnh là điều cơ bản và quan trọng nhất. Bởi vì từ đó ta có thể đánh giá được năng suất của cây, tìm ra được những mầm bệnh từ hình ảnh trái cây được tách ra từ ảnh gốc, …

Áp dụng vào thực tiễn tình sản xuất và bán trái cây của người dân miền Tây ở khu vực phía Nam Việt Nam. Do trái cây ở trên cây thường rất khó đếm được hết, người nông dân thường bán cả cây cho thương lái. Điều này làm cho việc mua bán trở nên không dễ dàng và có thể dẫn đến lỗ vốn cho người dân. Vì vậy để tìm lời giải cho vấn đề đánh giá năng suất cây trồng, đặc biệt là ở trường hợp trên, bằng mong muốn tìm hiểu, khám phá, phát triển khả năng nghiên cứu và đóng góp cho xã hội, nhóm quyết định thực hiện luận văn này giải quyết vấn đề cơ bản nhất cho việc đánh giá năng suất, đó là nhận diện trái cây trong hình ảnh cây ăn trái. Nhóm quyết định áp dụng kĩ thuật học sâu từ đó đưa ra hướng giải quyết cho bài toán nhận diện vật thể, cụ thể ở đây là nhận diện trái cây. Nhóm sử dụng giải thuật Faster R-CNN - một giải thuật nhận diện vật thể với độ chính xác cao để tiến hành nhận diện trái cây trong ảnh, rồi sau đó sử dụng một mạng tích chập để phân loại những trái cây nhận diện được là đúng hay sai để tăng độ chính xác của mô hình giải thuật.

\section{Những nghiên cứu liên quan}
Nhận diện vật thể không phải là một chủ đề mới trong lĩnh vực thị giác máy tính. Mô hình mạng nơ-ron tích chập (Convolutional Neural Network) đã mở ra nhiều hướng đi cho bài toán học có giám sát (Supervised Learning) và đã chứng minh sức mạnh của nó đối với dữ liệu huấn luyện lớn. Điểm mạnh của mô hình mạng tích chập là có thể được sử dụng để để huấn luyện một bộ xử lí “end to end”, nghĩa là nó có thể nhận dữ liệu đầu vào dưới dạng gốc như là một bức ảnh và đưa ra được kết quả phân loại của bức ảnh đó. Mặc dù tốt như vậy nhưng nó cũng tồn tại một điểm yếu lớn, đó chính là cần một lượng dữ liệu đầu vào lớn để được vào huấn luyện cho bộ học và việc gán nhãn cho dữ liệu huấn luyện rất mất thời gian và công sức nếu muốn có được một tập dữ liệu đa dạng và chính xác. Vì vậy việc chuẩn bị, sàn lọc, xử lí dữ liệu đầu vào là rất cần thiết đối với một giải thuật dựa trên mạng nơ-ron tích chập.


Nhận diện trái cây không phải là một chủ đề mới, đã có rất nhiều nghiên cứu liên quan về bài toán này \cite{bargoti2017image} \cite{sa2016deepfruits}. Một trong những phương pháp nhận diện trái cây đã được đề xuất trước đây trong bài báo của Cohen \cite{cohen2010estimation}. Hình ảnh họ sử dụng được chụp bởi camera theo chuẩn màu CCD. Đầu tiên, họ dùng một bộ phân loại K-nearest-neighbors (KNN) để xác định xem những điểm ảnh (pixel) nào là "táo" và những điểm ảnh nào là "không-phải-táo", những vật thể che mất quả táo như cành, lá được đánh dấu lại để loại bỏ ra khỏi quá trình huấn luyện. Sau đó đánh dấu bề mặt  của quả táo bằng cách cho nhận diện những vùng mà họ gọi là "seed area". Seed area là tập hợp những điểm ảnh có khả năng cao là táo. Sau đó những vùng seed area này được mở rộng ra để liên kết những vùng ảnh quá sáng hoặc quá tối nằm giữa hai seed area, từ đó tạo thành một vùng seed area hoàn chỉnh. Cuối cùng, họ phân tích các hình dạng của seed area và khoanh vùng được quả táo từ những đường nét của mỗi seed area. Tuy nhiên phương pháp này của họ vẫn chưa hiệu quả khi quả có quá nhiều hình dạng và màu khác nhau, nó cờn bị ảnh hưởng mạnh bởi độ sáng, bóng râm.

Nhiều nghiên cứu đều cho thấy vấn đề của quá trình nhận diện vật thể đó chính là công tác phân đoạn (segmentation), phải phân biệt rõ giữa vùng có vật thể và vùng nền chứa vật thể. Nhóm của Yamamoto \cite{yamamoto2014plant} đã sử dụng phương pháp phân đoạn hình ảnh dựa trên màu sắc để áp dụng cho học các đặc trưng trên ảnh. Mạng nơ-ron tích chập cũng có ưu điểm rõ ràng, đó chính là không cần phải trích xuất đặc trưng một cách thủ công. Các đặc trưng được trích xuất thông qua mạng tích chập được sử dụng vào quá trình nhận diện ảnh, có thể phân tích hình ảnh để lấy những đặc trưng low-level để giảm kích thước không gian nhận diện nhằm xác định vùng quan tâm (Region of Interests - RoIs) đồng thời khai thác đặc trưng high-level áp dụng vào quá trình phân loại.

Mô hình mạng R-CNN (Region based Convolutional Neural Network) cũng được đề ra nhằm giải quyết bài toán nhận diện. Những vùng quan tâm (RoIs) được tạo ra từ giải thuật Selective Search, sau đó được đưa qua mạng tích chập để được phân loại, đồng thời nó còn được sử dụng để tính toán hồi quy tìm ra bounding box. Faster R-CNN là mô hình tích hợp giữa các công việc là tìm vùng quan tâm, phân loại vật thể và tính toán hồi quy để tìm ra bounding box. Nhờ như vậy nên việc nhận diện vật thể trở nên nhanh hơn và hiệu quả hơn. Không những vậy, mô hình Faster R-CNN cho thấy một kết quả khả quan có thể áp dụng vào thực tế cho bài toán nhận diện trái cây \cite{bargoti2017deep}.

Tuy nhiên, đa số các nghiên cứu hiện nay đều hướng tới kết quả nhận diện được nhiều loại trái cây, đa phần là những loại trái cây ở khu vực nước ngoài, vì vậy việc áp dụng với các loại trái cây ở Việt Nam cũng rất cần được quan tâm. Do đó nhóm sẽ tập trung vào nhận diện, đánh giá những trái có tính chất như vậy, tiêu biểu là trái bưởi. Bài báo cáo này trình bày giai đoạn đầu tiên trong việc Xác định năng suất cây trồng bằng mạng học sâu, đó chính là nhận diện vị trí và phân loại trái cây. Để đạt được kết quả tốt nhất, nhóm thống nhất chọn mô hình mạng tích chập để dễ dành trích xuất đặc trưng từ  dữ liệu hình ảnh và framework Faster R-CNN, đã được cải tiến để giảm thiểu tối đa chi phí với một mạng rất sâu.

Ở phần sau, nhóm sẽ trình bày những nội dung sau:
\begin{itemize}
	\item Chương 2: Cơ sở lí thuyết
	\item Chương 3: Giải pháp đề xướng
	\item Chương 4: Ứng dụng Faster R-CNN vào trong nhận diện trái bưởi
	\item Chương 5: Kết luận và hướng phát triển, trình bày thêm về bộ phân loại áp dụng vào mô hình
\end{itemize} 

\section{Những đóng góp về mặt khoa học, thực tiễn của luận án}
\subsection{Về mặt khoa học}
Đóng góp một phương án để cải thiện độ chính xác, tăng số trái nhận diện được trong một ảnh, bằng cách kết hợp với một bộ phân loại nhằm phân biệt trái cây với nền, như vậy số vật thể phân loại sai sẽ giảm, đồng thời tỉ lệ nhận diện đúng số lượng trái sẽ tăng lên.
\subsection{Về mặt thực tiễn}
\begin{itemize}
	\item Đề ra một giải pháp để giải quyết bài toán nhận diện trái cây trong nông nghiệp, phục vụ cho nhu cầu thu hoạch tự động hoặc vẽ biểu đồ năng suất
	\item Đóng góp tập dữ liệu trái bưởi có độ chính xác cao, có thể dùng để tham khảo và sử dụng
\end{itemize}
